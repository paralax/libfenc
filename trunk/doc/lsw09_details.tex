\section{Lewko-Sahai-Waters KP-ABE \protect\cite{lsw09}}

\newcommand{\zkip}{{\textsc{zkip}}}
\newcommand{\vci}{{\textsc{vci}}}
\newcommand{\fip}{{\textsc{fip}}}
\newcommand{\start}{{\small{START}}}
\newcommand{\transcript}{{\small{TRANSCRIPT}}}
%\newcommand{\adv}{{\textbf{Adv}}}
\newcommand{\expe}{{\textbf{Exp}}}
%\newcommand{\hash}{$\mathcal{H}$}
\newcommand{\pe}{$\mathcal{PE}$}
%\newcommand{\s}{$\mathcal{SS}$}
\newcommand{\enc}{$\mathcal{E}$}
\newcommand{\dec}{$\mathcal{D}$}
\newcommand{\key}{$\mathcal{K}$}
%\newcommand{\G}{\ensuremath{\mathbb{G}}}
\newcommand{\GT}{\ensuremath{\mathbb{G}_T}}
%\newcommand{\Zp}{\ensuremath{\mathbb{Z}_p}}
%\newcommand{\G}{$\mathcal{G}$}
%\newcommand{\GT}{$\mathcal{G}_T$}
\newcommand{\univ}{\mathcal{U}}
\newcommand{\zz}{\mathbb{Z}}
\newcommand{\parent}{\mathrm{parent}}
\newcommand{\sons}{\mathrm{CHILD}}
\newcommand{\ind}{\mathrm{index}}
\newcommand{\att}{\mathrm{att}}
\newcommand{\ghat}{e(g,g)}
%\newcommand{\egg12}{$e(g_1,g_2)}
%\newcommand{\egg2}{e(g,g_2)}
\newcommand{\msp}{\mathrm{MSP}}
\newcommand{\PK}{params}
\newcommand{\genOne}{g}
\newcommand{\genTwo}{\bar{g}}
\newcommand{\hOne}{h}
\newcommand{\hTwo}{\bar{h}}

%\newcommand{\brent}[1]{\texttt{[brent: #1]}}
%\newcommand{\anote}[1]{\texttt{[Amit: #1]}}
%\newcommand{\comment}[1]{}
\newcommand{\MK}{\ensuremath{\textrm{MK}}}
%\newcommand{\PK}{\ensuremath{\textrm{PK}}}


\newcommand{\Decnode}{\ensuremath{\mathrm{DecryptNode}}}
\newcommand{\PolySat}{\ensuremath{\mathrm{PolySat}}}
\newcommand{\PolyUnsat}{\ensuremath{\mathrm{PolyUnsat}}}

This section documents some of the implementation-specific decisions made in implementing the Lewko-Sahai-Waters scheme.  In particular, the original scheme description specifies the use of symmetric groups; in our implementation we are required to specify which elements belong to $\gOne$ and $\gTwo$.  

The following quotes the original scheme description from \cite{lsw09} with minor modifications.  All group elements marked normally ({\it e.g.,} $g$) are in $\gOne$, while those marked with a bar ($\genTwo$) are in $\gTwo$.

\paragraph{Setup.}
The setup algorithm chooses generators $\genOne \in \gOne, \genTwo \in \gTwo$ and picks random exponents
$z, \alpha', \alpha'', b \in \Zp$. We define $\hOne = \genOne^z \in \gOne, \hTwo = \genTwo^z \in \gTwo, \alpha = \alpha' \cdot \alpha''$, $\genOne_1 = \genOne^{\alpha'}$ and $\genOne_2 = g^{\alpha''}$. The public parameters are published as the following, where $H$ is a random oracle that outputs elements of the elliptic curve group:
\[
\PK=( g,
g^{b},
g^{b^{2}},
h^{b},
e(g,g)^{\alpha}, H(\cdot) ).
\]
The authority keeps $(\alpha', \alpha'', b)$ as the master key $\MK$.


%Two secrets $\alpha, \beta$ are chosen uniformly at random from $\zz_p$,
%and we denote $g_1 = g^\alpha$ and $g_2 = g^\beta$.
%In addition, two polynomials $h(x)$ and $q(x)$ of degree $d$ are chosen
% at random subject to the constraint that $q(0)=\beta$. (There is no constraint on $h(x)$.)
%The public parameters $\PK$ are $(g, g_1;\mbox{  } g_2=g^{q(0)}, g^{q(1)}, g^{q(2)}, \dots, g^{q(d)};\mbox{  }$
%$g^{h(0)}, g^{h(1)}, \dots, g^{h(d)})$.  The master key $\MK$ is $\alpha$.
%
%These public parameters define two publicly computable functions $T, V : \zz_p \rightarrow \G$.
%The function $T(x)$
%maps to $g_2^{x^d}\cdot g^{h(x)}$, and the function $V(x)$ maps to $g^{q(x)}$.
% Note that both $g^{h(x)}$ and $g^{q(x)}$ can be
%evaluated from the public parameters by interpolation in the exponent. (For further details on how to do this using
%Lagrange coefficients, see, e.g.,~\cite{sw05,GPSW06}.)

\paragraph{Encryption} ($M,\gamma,\PK$).
To encrypt a
message $M \in \GT$ under a set of $d$ attributes $\gamma \subset \zz_p^*$,
choose a random value $s\in\zz_p$, and choose a random set of $d$ values $\{s_x\}_{x \in \gamma}$
such that $s= \sum_{x \in \gamma} s_x$. Output the ciphertext
as
\begin{eqnarray*}
E = (
\gamma, E^{(1)} = M e(g, g)^{\alpha \cdot s},
E^{(2)} = g^s,
\{E^{(3)}_x = H(x)^s\}_{x\in\gamma}, \\
\{E^{(4)}_x = g^{b\cdot s_x}\}_{x\in\gamma},
\{E^{(5)}_x = g^{b^2\cdot s_x x} h^{b \cdot s_x}\}_{x\in\gamma}
)
\end{eqnarray*}
%Note that
%$g^{q(i)}$ can be computed through interpolation in the exponent using the public parameters, and then $g^{sq(i)}$ can be computed by raising $g^{q(i)}$ to the $s$ power.

\paragraph{Key Generation} ($\mathbb{\tilde{A}},\MK,\PK$).
This
algorithm outputs a key that enables the user to decrypt an encrypted message \emph{only}
 if the attributes of that ciphertext
satisfy the access structure $\mathbb{\tilde{A}}$.  We require that the
access structure $\mathbb{\tilde{A}}$ is $NM(\mathbb{A})$ for some monotonic access structure $\mathbb{A}$, (see~\cite{OSW07} for a definition of the $NM(\cdot)$ operator)
over a set $\mathcal{P}$ of attributes, associated with a linear secret-sharing scheme $\Pi$.
First, we apply the linear secret-sharing mechanism $\Pi$ to obtain shares $\{\lambda_i\}$ of the
secret $\alpha'$.  We denote the
party corresponding to the share $\lambda_i$ as $\breve{x}_i \in \mathcal{P}$,
where $x_i$
is the attribute underlying $\breve{x}_i$.  Note that $\breve{x}_i$
can be primed (negated) or unprimed (non negated).
For each $i$, we also choose a random value $r_i\in\zz_p$.

The private key $D$ will consist of the following group elements:
For every $i$ such that $\breve{x}_i$ is \emph{not} primed (i.e., is a non-negated
attribute), we have $$D_i =
(D^{(1)}_i = g_2^{\lambda_i} \cdot H(x_i)^{r_i},
D^{(2)}_i = g^{r_i})$$
For every $i$ such that $\breve{x}_i$ is primed (i.e., is a negated attribute), we have
$$D_i = (
D^{(3)}_i = g_2^{\lambda_i}g^{b^2r_i},
D^{(4)}_i = g^{r_ibx_i}h^{r_i},
D^{(5)}_i = g^{-r_i})$$  The key $D$ consists of $D_i$ for all shares $i$.


\paragraph{Decryption} ($E,D$).
Given a ciphertext $E$ and a decryption key $D$, the following procedure is executed:
 (All notation here is taken from the above descriptions of $E$ and $D$, unless the
notation is introduced below.)
First, the key holder checks if $\gamma \in \mathbb{\tilde{A}}$ (we assume that
 this can be checked efficiently).
If not, the output is $\bot$.
If $\gamma \in \mathbb{\tilde{A}}$, then we recall that $\mathbb{\tilde{A}} = NM(\mathbb{A})$,
where $\mathbb{A}$ is an access structure, over a
set of parties $\mathcal{P}$, for a
linear secret sharing-scheme $\Pi$.
Denote $\gamma' = N(\gamma) \in \mathbb{A}$,
and let $I = \{i: \breve{x}_i \in \gamma' \}$.
Since $\gamma'$ is authorized, an efficient procedure associated with the
linear secret-sharing scheme yields a set of coefficients
$\Omega = \{\omega_i\}_{i \in I}$ such that
$\sum_{i \in I} \omega_i \lambda_i = \alpha$.
(Note, however, that these $\lambda_i$ are not known to the decryption procedure, so neither is $\alpha$.)

For every positive (non negated) attribute $\breve{x}_i \in \gamma'$
(so $x_i \in \gamma$), the decryption procedure computes the following:
\begin{eqnarray*}
Z_i &=& e\left(D^{(1)}_i, E^{(2)}\right) / e\left(D^{(2)}_i, E^{(3)}_i\right)\\
&=& e\left(g_2^{\lambda_i} \cdot H(x_i)^{r_i}, g^s\right) /
e\left(g^{r_i}, H(x)^s\right) \\
&=& e\left(g,g_2\right)^{s \lambda_i}
\end{eqnarray*}

For every negated attribute $\breve{x}_i \in \gamma'$
(so $x_i \notin \gamma$), the decryption procedure computes the following, following a simple analogy to the basic revocation scheme:
%
%The decryption algorithm computes:
%\[
%\frac{e(C_0,D_0)}
%{
%e\left(D_1, \prod_{i=1}^{n}C_{i,1}^{1/(\ID-\ID_i)}\right)
%\cdot
%e\left(D_2, \prod_{i=1}^{n}C_{i,2}^{1/(\ID-\ID_i)}\right)
%}
%\]
%which gives us $e(g,g)^{\alpha s}$; this can immediately be used to recover the message $M$ from $C'$.
%Note that this computation is only defined
%if $\forall i \quad \ID \neq \ID_i$.
%
%We consider the set $\gamma_i = \gamma \cup \{x_i\}$.
%Note that $|\gamma_i| = d+1$ and recall that the degree of the polynomial
% $q$ underlying the function $V$ is $d$.  Using the points in $\gamma_i$
%as an interpolation set, compute Lagrangian coefficients
%$\{\sigma_x\}_{x \in \gamma_i}$ such that
%$\sum_{x \in \gamma_i} \sigma_x q(x) = q(0) = \beta$.
%Now, perform the following computation:
\begin{eqnarray*}
Z_i &=&
\frac{e\left(D^{(3)}_i, E^{(2)}\right)}
{
e\left(D^{(4)}_i,
\prod_{x \in \gamma} \left(E^{(4)}_x\right)^{1/(x_i-x)}\right)
\cdot
e\left(D^{(5)}_i,
\prod_{x \in \gamma} \left(E^{(5)}_x\right)^{1/(x_i-x)}\right)
} \\
&=&
%\frac{e\left(g_2^{\lambda_i + r_i}, g^s\right)}
%{
%e\left(
%g^{r_i},
%\prod_{x \in \gamma} \left( V(x)^s \right)^{\sigma_x}\right)
%\cdot
%e\left(V(x_i)^{r_i}, g^s\right)^{\sigma_{x_i}}
%}
%\\
%&=&
%\frac{
%e\left(g_2^{\lambda_i}, g^s\right)
%\cdot
%e\left(g_2^{r_i}, g^s\right)
%}
%{
%e\left(
%g^{r_i},
%g^{s \sum_{x \in \gamma} \sigma_x q(x)}
%\right)
%\cdot
%e\left(
%g^{r_i \sigma_{x_i} q(x_i)},
%g^s\right)
%}
%\\
%&=&
%\frac{
%e\left(g_2,g\right)^{s \lambda_i}
%\cdot
%e\left(g, g\right)^{r_i s \beta}
%}
%{
%e(g,g)^
%{r_i s \sum_{x \in \gamma'} \sigma_x q(x)}
%}
%\\
%&=&
e\left(g,g_2\right)^{s \lambda_i}
\end{eqnarray*}


Finally, the decryption is obtained by computing

$$
\frac{E^{(1)}}{\prod_{i \in I} Z_i^{\omega_i}}
=
\frac{M e(g,g)^{s \alpha}}{e(g,g_2)^{s \alpha'}} = M$$


\paragraph{Note on Efficiency and Use of Random Oracle Model.}